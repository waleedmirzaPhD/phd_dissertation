
The following movies can be accessed at \href{https://github.com/waleedmirzaPhD/movies_thesis.git}{https://github.com/waleedmirzaPhD/movies\_thesis.git} \cite{movies_github}

\bigskip	

\noindent{\bf Movie~3.1:} \quad Dynamics during wound healing for various degrees of anisotropic activity quantified by $\kappa$. The density and nematic fields are represented by colormaps. The direction and length of black arrows indicate the direction and magnitude of the velocity field $\bm{v}$. The direction and length of red segments indicate the average molecular orientation $\bm{n}$ and the nematic order $S$.

\bigskip	


\noindent{\bf Movie~3.2:}  \quad Effect of activity (quantified by the ratio between the active nematic length-scale $\ell_a$ and system size $\ell_0$)  on flows and defect structure and dynamics in a confined active nematic system.  The density and nematic fields are represented by colormaps. The direction and length of black arrows indicate the direction and magnitude of the velocity field $\bm{v}$. The direction and length of red segments indicate the average molecular orientation $\bm{n}$ and the nematic order $S$.	

\bigskip	

\noindent{\bf Movie~3.3:} \quad	 Low Reynolds number active turbulence at a high activity (quantified by the ratio between the active nematic length-scale $\ell_a$ and system size $\ell_0$).  The density and nematic fields are represented by colormaps. The direction and length of black arrows indicate the direction and magnitude of the velocity field $\bm{v}$. The direction and length of red segments indicate the average molecular orientation $\bm{n}$ and the nematic order $S$.

\bigskip	

\noindent{\bf Movie~4.1:} Pattern formation in an active gel model not accounting for nematic order.

\bigskip

\noindent{\bf Movie~4.2:} Pattern formation for different values of the tension anisotropy coefficient $\kappa$, leading to (I) fibrillar patterns for $\kappa <0$, (II) tactoids for $\vert \kappa \vert \approx 0$, and (III) sarcomeric patterns for  $\kappa >0$.

\bigskip

\noindent{\bf Movie~4.3:} Effect of flow-alignment coupling coefficient $\beta$ on pattern formation for positive and negative tension anisotropy coefficient $\kappa$. Flow-alignment favors fibrillar patterns and disfavors sarcomeric patterns. 

\bigskip
\noindent{\bf Movie~4.4:} Emergence of active turbulence in fibrillar patterns as the magnitude of $\kappa$ increases.

\bigskip
\noindent{\bf Movie~4.5:} Alignment of fibrillar patterns (top), diversity of fibrillar network structure and dynamics as mechanical interaction between dense bundles (bottom). Examples correspond to those in Fig.~\ref{fig4.3}.

\bigskip
\noindent{\bf Movie~4.6:} Illustration of simulation protocol to estimate tension anisotropy from discrete network simulations for an isotropic (left) and anisotropic (right) representative volume element. The phases are (1) alignment of filament to reach desired nematic order, (2) addition of boundary anchors (green dots) to prevent network collapse and to measure tension, (3) addition of crosslinkers and motors (blue and red dots) to bring the system out-of-equilibrium while tracking tension along horizontal and vertical directions.

\bigskip
\noindent{\bf Movie~4.7:} Illustration of simulation protocol to estimate the generalized active tension conjugate to nematic order for an isotropic (left) and anisotropic (right) representative volume element with periodic boundary conditions. The phases are (1) alignment of filament to reach desired nematic order, (2) addition of crosslinkers and motors (blue and red dots) to bring the system out-of-equilibrium while tracking the average nematic order of the representative volume element.




\bigskip	

\noindent{\bf Movie~7.1:} \quad Cell division resulting from an enhanced activity band at the equator. Heat maps at the top represent the nematic order parameter $S$ and the orientation of line segments, whose length is proportional to $S$, represents the average molecular orientation $\bm{n}$.

\bigskip

\noindent{\bf Movie~7.2:} \quad Above a threshold value of uniform isotropic activity $\lambda$ and anisotropic activity $\kappa=1.5$, a surface of the cytoskeleton with a small perturbation in density at the equator destabilizes. A new stable state consists of bidirectional flows converging at the equator resulting in the division of the surface.

\bigskip

\noindent{\bf Movie~7.3:} \quad Above a threshold value of uniform isotropic activity $\lambda$ and anisotropic activity $\kappa=0$, a surface of the cytoskeleton with a small perturbation in density at the equator destabilizes. A new stable state consists of bidirectional flows converging at the equator resulting in pseudocleavage. On long-time scales, a secondary instability triggers resulting in a bidirectional flow transitioning to a unidirectional flow of actin which renders the surface motile.  

\bigskip

\noindent{\bf Movie~7.4:} \quad Above a threshold value of activity $\lambda$, a uniform surface of cytoskeleton destabilizes resulting in a self-organized asymmetrical bidirectional flow of actin. The bidirectional flow converges in the rear of the cell resulting in the self-organization of a contractile ring that pinches the cells at the back and renders the surface motile because of its asymmetry.




\bigskip

\noindent{\bf Movie~7.5} \quad Self-organized cell motility at $\kappa=0,0.75,0.875$  and $1.$



\bigskip

\noindent{\bf Movies~8.1-8.4:} \quad Relaxation dynamics of ellipsoidal surfaces with different aspect ratios driven by nematic and Helfrich bending free energy potentials.

\bigskip
\noindent{\bf Movie~8.5:} \quad Relaxation dynamics of a deflated passive nematic vesicle driven by nematic and Helfrich bending free energy potentials.

\bigskip
\noindent{\bf Movie~8.6:} \quad Dynamics of an active vesicle periodically oscillating between low-energy tetrahedral states and high-energy planar states.

\bigskip

\noindent{\bf Movie~9.1:} \quad Self-organization of a pulsating flow of transverse nematic bundles in a circular geometry.


\bigskip

\noindent{\bf Movie~9.2:} \quad Self-organization and elongation of radial nematic bundles in a circular geometry.



\newpage











