

\begin{Abstract}
	The structure and dynamics of important biological systems, ranging from cytoskeletal gels to tissues, are controlled by an interplay between activity, dissipation, nematic order, density, and geometry. In particular, the actin cytoskeleton is remarkably adaptable and versatile and self-organizes into a variety of architectures, notably dense nematic structures. Examples of such structures are contractile rings to accomplish cell division and wound healing, or stress fibers  playing an essential role in cell spreading, force generation and migration. However, the mechanisms leading to dense nematic bundles remain poorly understood and to our knowledge no prior theory explains their mesoscale self-assembly. Previous hydrodynamic approaches to active nematic systems have focused on incompressible liquid crystals. However, this models are not pertinent to actin gels, which are highly porous and exhibit very large density variations controlling activity gradients and active flows. Furthermore, porous actin networks can exhibit extended isotropic states without defects, and nematic order is presumably induced by flows and active mechanisms rather than by crowding. Finally, little attention has been paid to the formulation of active nematics on deformable surfaces and to a general computational framework to address such complex problems. Finally, we lack a unified understanding of actin network polymorphism, and more specifically of the physical mechanisms that enable a single active material to adopt very different network architectures with distinct cellular functions. 

To address these challenges, we develop a variety of theoretical and computational models for active nematic gels. In part I of the thesis, we focus on active nematic gels confined to a plane. We develop a transparent modeling
	framework for density-dependent active nemato-hydrodynamics based on Onsager’s variational formalism, according to which the dynamics result from a competition between free-energy release, dissipation and activity. We also 	develop a numerical finite element method based on a time-incremental version of Onsager's formalism, which inherits the nonlinear stability and thermodynamic consistency of the continuum principle. We use this numerical approach to study the assembly of a contractile ring during wound healing and the defect dynamics in a confined colony of contractile cells. We next study the active self-assembly of nematic patterns from a uniform and quiescent gel using linearized theory and nonlinear simulations. We establish the conditions
	for nematic bundle formation and how active gel parameters control the architecture, 
	connectivity and dynamics of self-organized patterns of nematic bundles. Finally, using discrete
	network simulations, we substantiate the major requirements for active nematic self-organization according to our theory. We thus provide a framework to understand
	the emergence and dynamics of mesoscale nematic organizations in actin networks.
	
	In  part II of the thesis, we develop a general model for active nematic gels on deformable surfaces. Again, Onsager's nonlinear formalism provides a simple method to transparently develop complex models. The resulting system tightly couples shape dynamics, nematic dynamics, density dynamics, and interfacial hydrodynamics. We exercise this model  to understand the self-organization of the cytokinetic ring using an  axisymmetric numerical formulation. Finally, we provide a general 3D computational approach and apply it to an inextensible active liquid crystalline deformable surface. We examine the interplay between the motion of nematic defects and shape.  
\end{Abstract}

{\footnotesize
	\emph{Keywords}: active nematic gels, actomyosin cytoskeleton, shape dynamics, nematic bundles, nematic defects, Onsager's formalism, deformable surface, finite element method
}
\vfill 
\cleardoublepage
