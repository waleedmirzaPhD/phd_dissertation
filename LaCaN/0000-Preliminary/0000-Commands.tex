\newcommand{\flexocomp}{{\small{\texttt{FLEXOCOMP}}~}}

% ---- Setups -----
% \newcommand{\boldmathematics}[1]{\text{\boldmath$#1$}}
\newcommand{\toVect}[1]{{\boldsymbol{#1}}} % Vectors
\newcommand{\toMat}[1]{{\mathbf{#1}}}      % Matrices
%\DeclareMathAlphabet{\mathdutchcal}{U}{dutchcal}{m}{n}
%\SetMathAlphabet{\mathdutchcal}{bold}{U}{dutchcal}{b}{n}
%\DeclareMathAlphabet{\mathdutchbcal}{U}{dutchcal}{b}{n}

% ----- Diff ------
\newcommand{\dd}{\text{\,}\mathrm{d}}
\newcommand{\ds}{\mathrm{s}}

% --- Variables ---
\newcommand{\x}{\toVect{x}}
\newcommand{\X}{\toVect{X}}
\newcommand{\Displacement}{{\toVect{u}}}
\newcommand{\displacement}{{\toVect{v}}}
\newcommand{\Traction}{\toVect{t}}
\newcommand{\HighOrderTraction}{\toVect{r}}
\newcommand{\HighOrderJump}{\toVect{j}}

\newcommand{\displacementKnown}{\overline{u}}
\newcommand{\DisplacementKnown}{\toVect{\overline{u}}}
\newcommand{\partialndisplacementKnown}{\overline{v}}
\newcommand{\partialnDisplacementKnown}{\overline{\toVect{v}}}
\newcommand{\partialxDisplacementKnown}{\overline{\partial_x(u)}}
\newcommand{\npartialxDisplacementKnown}{\overline{n\partial_x(u)}}
\newcommand{\TractionKnown}{\toVect{\tractionKnown}}
\newcommand{\tractionKnown}{\overline{t}}
\newcommand{\HighOrderTractionKnown}{\toVect{\highOrderTractionKnown}}
\newcommand{\highOrderTractionKnown}{\overline{r}}
\newcommand{\HighOrderJumpKnown}{\toVect{\highOrderJumpKnown}}
\newcommand{\highOrderJumpKnown}{\overline{j}}
\newcommand{\phiKnown}{\overline{\phi}}
\newcommand{\wKnown}{\overline{w}}

\newcommand{\n}{\toVect{n}}
\newcommand{\m}{\toVect{m}}
\newcommand{\tv}{\toVect{s}}

% ----- NonLinear ----

\newcommand{\Stress}{\toVect{\stress}}
\newcommand{\stress}{\sigma}
\newcommand{\E}{{\toVect{E}}}
\newcommand{\D}{{\toVect{D}}}
\newcommand{\F}{{\toMat{F}}}
\newcommand{\ff}{{\widetilde{F}}}
\newcommand{\FF}{{\toMat{\ff}}}
\newcommand{\fff}{{\check{F}}}
\newcommand{\FFF}{{\toMat{\fff}}}
\newcommand{\C}{{\toMat{C}}}
\newcommand{\B}{{\toMat{B}}}
\newcommand{\CC}{{\widetilde{\toMat{C}}}}
\newcommand{\cc}{{\widetilde{C}}}
\newcommand{\CCC}{{\check{\toMat{C}}}}
\newcommand{\ccc}{{\check{C}}}
\newcommand{\GG}{\widetilde{\mathfrak{E}}}

\newcommand{\Pstar}{{\toVect{p}}^{\textbf{r}}}
\newcommand{\pstar}{{p^\text{r}}}
\renewcommand{\P}{{\toVect{P}}}
\newcommand{\p}{{\toVect{p}}}
\newcommand{\e}{{\toVect{e}}}
\renewcommand{\d}{{\toVect{d}}}

\renewcommand{\S}{{\toMat{S}}}
\newcommand{\SMW}{{\toMat{S}^\text{Maxwell}}}
\newcommand{\SFL}{{\toMat{S}^\text{Flexo}}}
\newcommand{\Smw}{{{S}^\text{Maxwell}}}
\newcommand{\smw}{{\boldsymbol{\sigma}^\text{Maxwell}}}
\newcommand{\SP}{{\widehat{\toMat{S}}}}
\renewcommand{\sp}{{\widehat{S}}}
\renewcommand{\SS}{{\widetilde{\toMat{S}}}}
\renewcommand{\ss}{{\widetilde{S}}}

\newcommand{\coup}{{\mathds{C}}}
\newcommand{\ccoup}{{\widetilde{\mathds{C}}}}

% ---- Tensors ----
\newcommand{\StrainKnown}{\toVect{\strainKnown}}
\newcommand{\strainKnown}{\overline\strain}
\newcommand{\Strain}{\toVect{\strain}}
\newcommand{\strain}{\varepsilon}
\newcommand{\Pol}{\toVect{P}}
\newcommand{\CauchyStress}{\toVect{\cauchyStress}}
\newcommand{\cauchyStress}{\widehat{\sigma}}
\newcommand{\HyperStress}{\toVect{\hyperStress}}
\newcommand{\hyperStress}{\widetilde{\sigma}}
\newcommand{\ElectricDisp}{\toVect{\electricDisp}}
\newcommand{\electricDisp}{\widehat{D}}
\newcommand{\hyperelectricDisp}{\widetilde{D}}
\newcommand{\hyperElectricDisp}{\toVect{\hyperelectricDisp}}
\newcommand{\ShapeOperator}{\toMat{\shapeOperator}}
\newcommand{\shapeOperator}{S}
\newcommand{\Projector}{\toMat{\projector}}
\newcommand{\projector}{P}
\newcommand{\curvatureProjector}{\tilde{N}}
\newcommand{\CurvatureProjector}{\toMat{\curvatureProjector}}

% --- Material ----
\newcommand{\Piezo}{\toVect{\piezo}}
\newcommand{\piezo}{e}
\newcommand{\Flexo}{\toVect{\flexo}}
\newcommand{\flexo}{\mu}
\newcommand{\Flexocoup}{\toVect{\flexocoup}}
\newcommand{\flexocoup}{f}
\newcommand{\Dielec}{\toVect{\dielec}}
\newcommand{\dielec}{\kappa}
\newcommand{\Elast}{\toMat{c}}
\newcommand{\elast}{{c}}
\newcommand{\StrGr}{\toMat{\strGr}}
\newcommand{\strGr}{h}
\newcommand{\grdie}{M}
\newcommand{\Grdie}{\toMat{\grdie}}

% --- Quantum ---
\newcommand{\hamiltonian}{\widehat{H}}
\newcommand{\planck}{\hslash}
\newcommand{\radialpolarization}{{p_\text{r}}}
\newcommand{\bohr}{\textrm{~bohr}}

% --- Operators ---
\newcommand{\divergence}{\nabla\text{\!}\cdot\text{\!}}
\newcommand{\gradient}{\nabla}
\newcommand{\symmetricGradient}{\nabla^\text{sym}}
\newcommand{\surfaceGradient}{\nabla^{S}}
\newcommand{\normalGradient}{\nabla^{N}}
\newcommand{\surfaceDivergence}{\nabla^S\text{\!}\text{\!}\cdot\text{\!}}
\newcommand{\normalDivergence}{\nabla^N\text{\!}\text{\!}\cdot\text{\!}}
\newcommand{\jump}[2][]{\ifstrempty{#1}
    {{\mathchoice
    {{\left[\mkern-3mu{\left[ #2 \right]}\mkern-3mu\right]}} % \displaystyle
    {{\left[\!\left[ #2 \right]\!\right]}} % \textstyle
    {{\left[\!\left[ #2 \right]\!\right]}} % \scriptstyle
    {{\left[\!\left[ #2 \right]\!\right]}} % \scriptscriptstyle
    }}
    {{\mathchoice
    {{\left[\mkern-3mu{\left[ #2 \right]}\mkern-3mu\right]}_{#1}} % \displaystyle
    {{\left[\!\left[ #2 \right]\!\right]}_{#1}} % \textstyle
    {{\left[\!\left[ #2 \right]\!\right]}_{#1}} % \scriptstyle
    {{\left[\!\left[ #2 \right]\!\right]}_{#1}} % \scriptscriptstyle
    }}
    }
\newcommand{\mean}[1]{\left\{\!\!\left\{ #1 \right\}\!\!\right\}}
\DeclareMathOperator{\Trace}{Tr}
\newcommand{\trace}[1]{\Trace(\,#1\,)}
\newcommand{\Log}[1]{\log(\,#1\,)}
\newcommand{\curl}{\nabla\times}
\DeclareMathOperator*{\argmin}{\arg\!\min}
\DeclareMathOperator*{\dist}{dist}
\DeclareMathOperator*{\argmax}{\arg\!\max}
\DeclareMathOperator*{\Supp}{supp}
\DeclareMathOperator*{\SYMM}{symm}
\newcommand{\symm}[2][]{\ifstrempty{#1}
	{\SYMM\left(#2\right)}
	{\SYMM_{#1}\left(#2\right)}}
\newcommand{\id}{\updelta}
\newcommand{\supp}[1]{\Supp\Big(#1\Big)}
\newcommand{\var}[2][]{\ifstrempty{#1}
	{{\delta #2}}
	{{\delta\hspace{-0.1em}_{#1} #2}}}
\newcommand{\vvar}[2][]{\ifstrempty{#1}
	{{\delta^2 #2}}
	{{\delta^2\hspace{-0.1em}_{#1} #2}}}
\newcommand{\norm}[1]{{\left\lVert #1 \right\rVert}}

% ----- Text ------
\newcommand{\code}[1]{\texttt{#1}}
\newcommand{\eq}{Eq.~}
\newcommand{\eqs}{Eqs.~}
\newcommand{\fig}{Fig.~}
\newcommand{\figs}{Figs.~}
\newcommand{\etal}{et~al.~}
\newcommand{\ie}{i.e.~}
\newcommand{\eg}{e.g.~}
\newcommand{\cf}{cf.~}
\renewcommand{\sec}{Section~}

% ----- Units ------
\DeclareSIUnit \uJ  { \micro \joule }
\sisetup{product-units = power,mode = math,per-mode=symbol}
\newcommand{\um}{~\si{\um}}
\newcommand\hmmax{0}
\newcommand\bmmax{0}





\newcommand\dcirclearrowleft{\mathrel{%
  \ensurestackMath{\stackengine{0pt}{\circlearrowleft}{\cdot}{O}{c}{F}{T}{L}}%
}}
\newcommand{\marino}[1]{\textcolor[rgb]{0.7,0.2,0.5}{#1}}
\newcommand{\torres}[1]{\textcolor[rgb]{0.2,0.7,0.5}{#1}}
\newcommand{\mirzawd}[1]{\textcolor[rgb]{0, 0.125, 0.376}{#1}}
\newcommand{\tensorm}{%
	% Uncomment below to adjust the gap between the dots
	%\setstackgap{S}{0.6ex}%
	\mathrel{\Shortstack{{.} {.} {.}}}}
\def\onedot{$\mathsurround0pt\ldotp$}
\def\cddot{% two dots stacked vertically
	\mathbin{\vcenter{\baselineskip.67ex
			\hbox{\onedot}\hbox{\onedot}}%
}}%
\def\cdddot#1{% three dots 
	\mathbin{\vcenter{\baselineskip.67ex
			\hbox{\onedot}\hbox{\onedot}\hbox{\onedot}%
	}}%
}
\makeindex
\makeatletter	
\newcommand*{\rom}[1]{\expandafter\@slowromancap\romannumeral #1@}
\DeclareMathAlphabet{\mathpzc}{OT1}{pzc}{m}{it}
\makeatletter %% <- make @ usable in macro names
\newcommand*\superimpose[2]{%
	\ooalign{$\m@th#1\@firstoftwo#2$\cr
		\hidewidth$\m@th#1\@secondoftwo#2$\hidewidth}%
}
\makeatother
\newcommand*\threedotsord{\,\mathpalette\superimpose{{\mathop:}{\cdot}}\,}
\renewcommand\thesection{\arabic{section}}
\newcommand{\mi}{\mathrm{i}} %% roman "i"

